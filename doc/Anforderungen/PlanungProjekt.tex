\documentclass{scrartcl}
\setcounter{tocdepth}{4}
\setcounter{secnumdepth}{4}
\usepackage[utf8]{inputenc}
\usepackage{graphicx} 
\usepackage{tabularx}
\usepackage{listings}
\usepackage{float}
\usepackage{xcolor}
\usepackage{scrpage2}
\usepackage{tikz}
\usepackage[top=3.5cm, bottom=2.5cm, left=2.5cm, right=2.5cm,headheight=110pt]{geometry}

\title{Anperi}
\subtitle{Projekt SS 2018\\ Projektbeschreibung \\  SS2018}
\author{Jannes Peters - 590252 \\ Adrian Kurth - 590289 \\ Jesse Nis Arff - 590245}
\date{\today}

\begin{document}

\maketitle
\newpage
\renewcommand\contentsname{Inhalt}
\tableofcontents{}
\newpage

\pagestyle{scrheadings}
\addtolength{\voffset}{-15pt}
\ihead{\textbf{Projekt SS 2018}}  
\chead{Projekt: Anperi \\ Dok.-Typ: Anforderungsliste}
\ohead{Datum: \today}
\setheadsepline{0.4pt}

\section{Beschreibung}
\textbf{Androidgeräte als PC-Peripherie (I/O-Controller)}\\
Die meisten PC-Nutzer nutzen auch mobile Endgeräte wie Smart­phones oder Tablets. Es existieren bereits Apps, die es ermöglichen, ein Tablet als weiteren (Touch-)Bildschirm an einem PC zu nutzen. Diese Apps sind jedoch in der Regel recht allgemein gehalten.\\
In diesem Projekt soll eine Schnittstelle in Form einer Programm­bibliothek (Java/C\#/C++) und einer Android-App als Gegenstück entwickelt werden. Dies soll es unterschiedlicher Software erlauben, durch individuelle Erweiterungen ein Tablet oder Smartphone als Anzeige- und Eingabe­medium zu nutzen und dabei gezielt auf die Anforderungen der zu bedienenden Software einzugehen.
\section{Komponenten}
\begin{itemize}
	\item Android-App
	\item C\#-Schnittstelle
	\item Java-Schnittstelle
	\item Verbindungsserver
\end{itemize}
\section{Anforderungen}
\subsection{Primäre Anforderungen}
\begin{itemize}
	\item dynamische GUI-Erstellung
	\item Senden/Empfangen von GUI-Events zwischen PC/Android-App
	\item erstmaliges Verbinden mit Verbindungs-Code, danach automatisch
	\item Kopplung von mehreren Android-Geräten mit dem PC
\end{itemize}
\subsection{Sekundäre Anforderungen}
\begin{itemize}
	\item Testprogramm, welches die Schnittstelle verwendet
\end{itemize}
\section{Zeitplan}
\begin{description}
	\item[06.04.2018] Erstellen der Anforderungen und des Zeitplans
	\item[20.04.2018] Prototyp für Verbindung zwischen den Geräten
	\item[04.05.2018] C\#- und Android-Schnittstelle
	\item[18.05.2018] dynamisches Erstellen der GUI
	\item[01.06.2018] Java-Schnittstelle
	\item[15.06.2018] Fertigstellung
\end{description}
\end{document}